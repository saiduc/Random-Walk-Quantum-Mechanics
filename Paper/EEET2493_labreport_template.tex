\documentclass[journal]{IEEEtran}

% *** CITATION PACKAGES ***
\usepackage[style=ieee]{biblatex} 
\bibliography{random_walk.bib}    %your file created using JabRef

% *** MATH PACKAGES ***
\usepackage{amsmath}

% *** PDF, URL AND HYPERLINK PACKAGES ***
\usepackage{url}
\hyphenation{op-tical net-works semi-conduc-tor}
\usepackage{graphicx}  %needed to include png, eps figures
\usepackage{float}  % used to fix location of images i.e.\begin{figure}[H]

\begin{document}

% paper title
\title{Using Random Walk Simulations to Calculate Ground State Energies in Quantum Physics}

% author names 
\author{Sai Pandian, ID: 29899923}% <-this % stops a space
        
% The report headers
\markboth{PHYS6017 Computer Techniques in Physics Report 1, April 2020}
{Shell \MakeLowercase{\textit{et al.}}: Bare Demo of IEEEtran.cls for IEEE Journals}

% make the title area
\maketitle

% As a general rule, do not put math, special symbols or citations
% in the abstract or keywords.
\begin{abstract}
Provide a summary of the session. What was done, 
what measurements were taken, brief methods, what calculations, brief conclusion.  The Abstract should be approximately 250 words or fewer, italicized, in 10-point Times (or Times Roman.) Please leave two spaces between the Abstract and the heading of your first section.
It should briefly summarize the essence of the paper and address the following areas without using specific subsection titles. Objective: Briefly state the problem or issue addressed, in language accessible to a general scientific audience. Technology or Method: Briefly summarize the technological innovation or method used to address the problem. Results: Provide a brief summary of the results and findings. Conclusions: Give brief concluding remarks on your outcomes. Detailed discussion of these aspects should be provided in the main body of the paper.
\end{abstract}

\begin{IEEEkeywords}
keywords, temperature, xxxx equation, etc.
\end{IEEEkeywords}

\section{Introduction}
% Here we have the typical use of a "W" for an initial drop letter
% and "RITE" in caps to complete the first word.
% You must have at least 2 lines in the paragraph with the drop letter
% (should never be an issue)

\IEEEPARstart{S}{ample} Text

\section{Theoretical Background}

Sample Text

\section{Method}

\begin{table}[!ht] %[H]
\centering
\label{table:Exps}
\begin{tabular}{ll}
Student &  Max Temperature \\ \hline
aabbbccc &  $35^{\circ}$   \\
eeeddd &   $54^{\circ}$ \\
eeeddd &   $54^{\circ}$ \\
\end{tabular}
\caption{Temperature measurements performed for session 1.}
\end{table}


\begin{figure}[H]%[!ht]
\begin {center}
\includegraphics[width=0.45\textwidth]{images/ecg.png}
\caption{Illustrations, graphs, and photographs may fit across both columns, if necessary. Your artwork must be in place in the article.}
\label{fig:ecg}
\end {center}
\end{figure}


\section{Results and Discussion}

Sample Text


\section{Conclusions}

Sample Text

% if have a single appendix:
%\appendix[Proof of the Zonklar Equations]
% or
%\appendix  % for no appendix heading
% do not use \section anymore after \appendix, only \section*
% is possibly needed

% use appendices with more than one appendix
% then use \section to start each appendix
% you must declare a \section before using any
% \subsection or using \label (\appendices by itself
% starts a section numbered zero.)
%



\appendices
\section{Hand calculations (or name your title for appendix subtitle)}

Sample Text

\section*{Acknowledgment}
The authors would like to thank...



% references section

% can use a bibliography generated by BibTeX as a .bbl file
% BibTeX documentation can be easily obtained at:
% http://mirror.ctan.org/biblio/bibtex/contrib/doc/
% The IEEEtran BibTeX style support page is at:
% http://www.michaelshell.org/tex/ieeetran/bibtex/
%\bibliographystyle{IEEEtran}
% argument is your BibTeX string definitions and bibliography database(s)
%\bibliography{IEEEabrv,../bib/paper}
%
% <OR> manually copy in the resultant .bbl file
% set second argument of \begin to the number of references
% (used to reserve space for the reference number labels box)

%use following command to generate the list of cited references

\printbibliography

% that's all folks
\end{document}


